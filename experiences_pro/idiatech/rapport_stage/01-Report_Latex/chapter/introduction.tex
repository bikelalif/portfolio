\section{idIA tech}\label{sec:enterprise}
Fondée en 2009, idIA Tech est une société française spécialisée dans le web-mining, évoluant dans le secteur des technologies de l’information et de l’Internet. C’est une société à taille humaine, donc très proche de ses salariés et ses clients.
\\
\\
Son siège est situé à Montpezat, en Occitanie mais elle dispose également de plusieurs annexes en France, notamment à Bourg-la-Reine.
\\
\\
idIA Tech concentre son activité sur l’extraction de données issues de catalogues fournisseurs, particulièrement dans le domaine du e-commerce, qui automatise le travail des e-commerçants en important des catalogues entier de produits. Dans un contexte où le e-commerce est en forte croissance, l’automatisation de l’import de catalogues constitue un avantage compétitif essentiel pour les clients d’idIA Tech. Pour ce faire, elle s’apppuie sur un crawleur dynamique et sur Grimport, un langage de programmation développé par leurs équipes facilitant l’automatisation de l’intégration de données dans des CMS tels que Prestashop.


\section{Département de travail}\label{sec:department}
idIA Tech étant une PME (petite moyenne entreprise), l’entreprise ne dispose pas de départements structurés comme dans les grandes organisations. L’ensemble des projets est porté par une équipe réduite de développeurs, et de stagiaires avec qui j’étais en étroite collaboration, sous contrôle de notre maitre de stage.

\section{Contexte / Problème}\label{sec:context}
Mon stage s’inscrivait dans ce double contexte : traiter des dossiers clients en développant des scripts Grimport pour l’import de catalogues et contribuer à l’amélioration technique de l’IDE
L’utilisation de Grimport facilite l’automatisation de l’import de catalogues fournisseurs, mais l’approche par crawler soulève plusieurs difficultés techniques au niveau du crawler.
\\
\\
Le travail sur le débogueur Grimport a présenté un autre type de difficultés, liées cette fois aux technologies utilisées vieillissante donc soulevait des problèmes de compatibilités. Celui ci bloquait les évolutions de l’IDE, ce qui freinait la modernisation de l’outil, et donc compliquait la conception des scripts Grimport.


\section{Objectifs du stage}\label{sec:objectives}
L’objectif principal de mon stage était de m’intégrer à l’équipe en tant qu’apprenti développeur Grimport. Pouvoir gérer des dossiers clients seul et répondre à leur cahier des charges .
Parallèlement, un objectif plus personnel m’a été confié : travailler sur l’amélioration de l’IDE Grimport. Résoudre le problème de débogueur qui bloquait la mise à jour de l’IDE.


\section{Principales contributions}
Au cours de mon stage, j’ai d’abord contribué en tant que développeur Grimport à la réalisation de plusieurs scripts destinés à différents clients. Ces scripts permettaient d’importer automatiquement leurs catalogues fournisseurs dans leurs CMS, en particulier Prestashop, et d’adapter les extractions aux besoins propres à chaque projet.
En parallèle, j’ai travaillé sur l’amélioration technique de l’IDE Grimport, en m’attaquant aux erreurs Java liées au code source Groovy modifié par mes prédécesseurs. Mon objectif était de parvenir à compiler ce code en .jar, de régler tous les problèmes de librairie et de dépendances que le projet débogueur contenait et de l’intégrer au projet, afin de restaurer le bon fonctionnement du débogueur. 

