
\begin{description}
  \item[API] (Application Programming Interface) Interface de programmation qui permet à deux applications de communiquer entre elles.
  
  \item[CMS] (Content Management System) Système de gestion de contenu permettant de créer et gérer des sites web, comme Prestashop.
  
  \item[Grimport] Langage et IDE propriétaire développé par idIA Tech pour automatiser l’import de catalogues fournisseurs dans les CMS.
  
  \item[IDE] (Integrated Development Environment) Environnement de développement intégré regroupant un éditeur de code, un compilateur et des outils de débogage.
  
  \item[Gradle] Outil de build et de gestion des dépendances pour les projets Java et Groovy.
  
  \item[Groovy] Langage de programmation orienté objet basé sur la JVM, offrant une syntaxe souple et dynamique proche de Java.
  
  \item[Jar] (Java ARchive) Format de fichier permettant de regrouper des classes Java compilées et leurs ressources dans une seule archive exécutable.
  
  \item[Captcha] (Completely Automated Public Turing test to tell Computers and Humans Apart) Test permettant de différencier un utilisateur humain d’un programme automatisé.
  
  \item[Fiddler] Outil d’analyse réseau permettant d’intercepter et d’examiner les requêtes HTTP/HTTPS.
  
  \item[Firefox Developer Edition] Version du navigateur Firefox destinée aux développeurs, intégrant des outils avancés d’inspection et de débogage.
  
  \item[Maven Repository] Dépôt public (mvnrepository.com) contenant des bibliothèques Java et Groovy sous forme de .jar.
  
  \item[Retro-engineering] Analyse d’un site web ou d’une application existante afin d’en comprendre la structure et d’automatiser des actions.
  
  \item[Plugin GDT] (Groovy Development Tools) Extension Eclipse permettant la prise en charge du langage Groovy dans l’IDE.
\end{description}
