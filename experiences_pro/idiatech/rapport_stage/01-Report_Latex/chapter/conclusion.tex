\section{Conclusion}
Ce stage m’a permis de travailler sur deux axes complémentaires.  
D’une part, le développement de scripts Grimport pour différents clients, automatisant l’import des catalogues fournisseurs dans Prestashop et améliorant ainsi l’efficacité du processus. Cela m'a permit de faire mes premiers pas dans la relation client, établir un cahier des charges et de la charge de travail demandée pour un projet précis.
\\
D’autre part, la reprise et la remise en état du débogueur de l’IDE Grimport, dont le code source Groovy/Java modifié était devenu incompilable, m'a sérieusement sortit de ma zone de confort et m'a pousser à une très grande réflexion et minutie dans tout ce que je modifiais. J'ai réellement pu prendre conscience de mes capacités.
\\
Ces deux volets m’ont conduit à mobiliser des compétences en web-mining, en rétro-ingénierie, en gestion de dépendances et en intégration d’outils dans un environnement complexe. 
\\
Le travail réalisé a permis de livrer des scripts robustes et de satisfaire les clients ainsi que de m'avanturer dans un projet java/groovy réellement complexe, ouvrant la voie à sa compilation future.

\section{Leçons retenues}
Ce stage a été une occasion d’apprentissage riche :
\begin{itemize}
  \item Renforcement des compétences techniques : développement de scripts d’extraction et d’intégration de données, gestion d’APIs externes (traduction, Captcha), manipulation d’outils comme Gradle, Fiddler, Firefox Developer Edition.
  \item Amélioration notable de ma capacité à m’adapter à différents langages en m’appuyant sur la documentation technique.
  \item Acquisition d’une méthodologie de diagnostic et de résolution de problèmes sur un projet existant et modifié par d’autres.
  \item Développement de l’autonomie et de la communication avec les clients.
  \item Compréhension des enjeux liés à la maintenance et à l’évolution d’un IDE interne dans une entreprise.
\end{itemize}

\section{Perspectives et étapes suivantes}
Plusieurs pistes peuvent être envisagées pour prolonger ce travail :
\begin{itemize}
  \item Finaliser la compilation complète du débogueur en identifiant ou en remplaçant les bibliothèques encore manquantes.
  \item Améliorer la documentation technique du débogueur pour faciliter sa prise en main par de futurs développeurs.
\end{itemize}
Ces perspectives permettront de consolider les acquis du stage et de poursuivre la modernisation de l’IDE Grimport tout en offrant des outils plus performants aux utilisateurs finaux.

