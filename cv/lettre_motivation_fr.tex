\documentclass[11pt,a4paper]{article}
\usepackage[utf8]{inputenc}
\usepackage[T1]{fontenc}
\usepackage[french]{babel}
\usepackage[margin=2cm]{geometry}
\usepackage{xcolor}
\usepackage{hyperref}
\usepackage{fontawesome5}
\usepackage{parskip}
\usepackage{setspace}

% Couleurs (identiques au CV)
\definecolor{primary}{RGB}{30, 64, 175}
\definecolor{secondary}{RGB}{71, 85, 105}
\definecolor{accent}{RGB}{37, 99, 235}

% Liens
\hypersetup{
    colorlinks=true,
    linkcolor=accent,
    urlcolor=accent
}

% Pas de numéros de page
\pagestyle{empty}

\begin{document}

% En-tête
\begin{center}
    {\Huge\bfseries\color{primary} Bilal KEFIF}\\[5pt]
    {\large\color{secondary} Étudiant Ingénieur en Génie Logiciel}\\[10pt]
    \faMapMarker*\ Paris, France \quad
    \faPhone\ +33 6 22 39 22 30 \quad
    \faEnvelope\ \href{mailto:bilalkefif243@gmail.com}{bilalkefif243@gmail.com}\\[5pt]
    \faLinkedin\ \href{https://www.linkedin.com/in/bilal-kefif-94400015a}{linkedin.com/in/bilal-kefif-94400015a} \quad
    \faGlobe\ \href{https://portfoliobilalkefif.netlify.app}{portfoliobilalkefif.netlify.app}
\end{center}

\vspace{1cm}

\begin{flushright}
    À l'attention du Responsable du Recrutement\\
    \today
\end{flushright}

\vspace{0.5cm}

\textbf{Objet : Candidature pour un stage de 3 à 4 mois en Cybersécurité / Développement Logiciel}

\vspace{0.5cm}

Madame, Monsieur,

Actuellement étudiant en deuxième année à l'\textbf{ENSIIE} (École Nationale Supérieure d'Informatique pour l'Industrie et l'Entreprise), je me spécialise en génie logiciel. Dans le cadre de ma formation, je réaliserai prochainement un échange académique à l'\textbf{Université Laval} (Québec) axé spécifiquement sur la \textbf{cybersécurité}. Je suis à la recherche d'un stage de 3 à 4 mois à compter de juin 2026 pour mettre en pratique mes compétences techniques et ma passion pour la sécurité des systèmes.

Mon parcours académique m'a permis d'acquérir des bases solides en développement (C, Java, Python) et en concepts fondamentaux de la sécurité (méthodes formelles, vérification logicielle). Mon semestre à l'Université Laval renforcera ces acquis avec des modules avancés tels que la sécurité des logiciels, la sécurité de l'IoT et la technologie Blockchain. Je suis particulièrement intéressé par l'opportunité d'appliquer ces connaissances théoriques à des problématiques réelles de sécurité ou de développement complexe au sein de votre entreprise.

Lors de mon précédent stage chez \textbf{idIA Tech}, j'ai développé une rigueur professionnelle et une capacité d'analyse approfondie. J'y ai notamment pratiqué l'automatisation de processus, des tâches nécessitant une compréhension fine du fonctionnement des applications web et des protocoles réseaux. J'ai également contribué à la restauration d'un débogueur d'IDE, démontrant ma capacité à découvrir et analyser un code complexe pour le corriger et l'optimiser.

Polyvalent et curieux, j'ai multiplié les projets techniques variés, allant du développement d'applications web (React/Flask) à la programmation système (Jeu de cartes en C, Éditeur de texte en OCaml). Cette diversité d'expériences témoigne de ma capacité à m'adapter rapidement à de nouvelles technologies et environnements.
Conscient que mon parcours étudiant peut présenter des lacunes face aux exigences professionnelles, je compense activement par une démarche d'auto-formation continue. Je me forme régulièrement via des plateformes spécialisées telles que TryHackMe et participe à des formations complémentaires en cybersécurité. Je suis réellement déterminé à mettre à profit cette passion profonde pour ces domaines.

Dynamique et motivé, je serais ravi de rejoindre vos équipes pour contribuer à vos projets tout en poursuivant mon apprentissage. Je me tiens à votre entière disposition pour un entretien afin de vous exposer plus en détail ma motivation.

En vous remerciant de l'attention que vous porterez à ma candidature, je vous prie d'agréer, Madame, Monsieur, l'expression de mes salutations distinguées.

\vspace{1cm}

\begin{flushright}
    \textbf{Bilal KEFIF}
\end{flushright}

\end{document}
